\subsubsection*{Desarrollo}
Explicar EG, resolucion de sistema y cmo funcina nuestra estrutura.

\par En primer lugar comenzamos planteando nuestra estructura como un diccionario DOK (Dictionary of Keys) -explicado en la diapositiva en clase- debido a una relativa facilidad a la hora de implementarla, sin embargo correr los primeros tests (los de la cátedra) nos encontramos con que esta implementación no cumplía nuestras expectativas en cuanto al tiempo de ejecución.

\par Al comenzar nuestra estructura fue un map, luego lo cambiamos por un unordered map en nuestra busqueda por bajar los tiempos de ejecución. Y finalmente terminamos volviendo a utilizar un map, solo que esta vez con iteradores, y fue con esta estructura que logramos correr el tp en tiempos que consideramos razonables.

\par EXPLICAR FUNCIONES BASICAS, MULTIPLICACION, SUMA DE MATRICES.

\par Luego, para implementar la Eliminación Gausseana lo llevamos a cabo siguiendo el método explicado en clase, con operaciónes de suma y multiplicación entre filas pero sin pivoteo entre las mismas.
EXPLICACION MAS DETALLADA DE EG

\par ACA EXPLICAR RESOLUCION DE SISTEMA 

