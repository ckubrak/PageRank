\subsubsection*{Desarrollo}

\par Las relaciones entre p\'aginas forman un grafo y resulta conveniente almacenarlos como una
matriz de uno y ceros, done el un uno en la poscici\'on $ij$ representa que la p\'agina $j$ apunta a la $i$.
A\'un as\'i esta implementaci\'ion tiene un problema: debe almacenar $n^2$ elementos, donde solo importan
$m$ (la cantidad de unos). En la pr\'actica, $m \ll n^2$ (ya que no todas las p\'aginas se relacionan
con todos). Es por esto que decidimos utilizar matrices \textit{ralas}: un tipo de matriz donde solo
importa donde hay unos, es decir, qu\'e p\'agina apunta a cu\'al.
\par Como primer acercamiento, evaluamos utilizar dos conocidos conocidos m\'etodos para el almacenamiento
de matrices ralas: CSR (Compressed Sparse Row) y CSC (Compressed Sparse Column). En el primer caso
nos encotramos con el problema de la dificultad de acceder a las colmunas, mientras que en el segundo,
las filas. consideramos de vital importancia poder acceder tanto a filas como columnas en un tiempo
razonable para operaciones tales como la multiplicación o la Eliminación Gausseana.
\par Luego de evaluar los requerimientos, tanto de complejidad como de espacio utilizado, optamos
por implementar una estructura h\'ibrida entre un DOK (Dictionary Of Keys) y una lista de listas. 
Utilizamos una estructura que consiste en un diccionario de diccionarios: en el primero almacenamos
todas las filas y en el segundo sus elementos.
\par Respecto a la implementaci\'on, nos encontramos frente a la decisión de utilizar un diccionario ordenado
implementado sobre una estructura autobalanceada (\verb|std::map|) o un diccionario desordenado
implementado sobre una tabla de hash (\verb|std::unordered_map|). Si bien este \'ulitmo permite
el acceso en $O(1)$ en promedio frente al acceso en $O(\log n)$ del diccionario ordenado, 
no resulta f\'acil iterar eficientemete por lo que terminamos decidi\'endonos por la versi\'on autobalanceada.
\par Para resolver el sistema \ref{ipwd} resulta pr\'actico y eficiente utilizar Eliminación Gausseana:

\begin{codebox}
\Procname{$\proc{Eliminación Gausseana}(A)$}
\li \For $k \gets 1$ \To $n-1$
    \Do
\li     \For $i \gets k+1$ \To $n$
            \Do
\li         \For $j \gets k+1$ \To $n$
                \Do
\li                 $a_{ij} \gets a_{ij} - a_{ik}a_{kj}$
                \End
            \End
        \End
\end{codebox}

Explicar EG, resolucion de sistema y cmo funcina nuestra estrutura.
\par EXPLICAR FUNCIONES BASICAS, MULTIPLICACION, SUMA DE MATRICES.

\par Luego, para implementar la Eliminación Gausseana lo llevamos a cabo siguiendo el método explicado en clase, con operaciónes de suma y multiplicación entre filas pero sin pivoteo entre las mismas.
EXPLICACION MAS DETALLADA DE EG

\par ACA EXPLICAR RESOLUCION DE SISTEMA 

\par 