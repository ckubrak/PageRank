\subsection*{An\'alisis cuantitativo}
\par A partir del an\'alisis cuantitativo, son varias las cosas que nos resulta importante destacar.
\par Tal como se puede apreciar en la Figura \ref{fig:sparcity} la sparcity afecta notablemente el rendimiento en cuanto a tiempo del algoritmo,
lo cual nos hace pensar que nuestra implementaci\'on funciona bien para matrices con $\delta \approx 0$ y no tanto con $\delta \approx 1$.
Es por esto que para casos donde la matriz no es rala, podr\'ia resultar m\'as eficiente una implementaci\'on distinta.
\par As\'i mismo, a partir de la Figura \ref{fig:proba} podemos tambi\'en notar una correlaci\'on con el tiempo. Consideramos que esto se debe
principalmente al hecho de que con menor $p$, menor es el m\'odulo del resultado de $pWD$, raz\'on por la cual en la ecuaci\'on \ref{ipwd} hay en
cuesti\'on n\'umeros m\'as peque\~os que por la precisi\'on dictada por $\varepsilon$ terminan desapareciendo.
\par En el \'ultimo caso, no pudimos establecer una relaci\'on entre el tiempo y el $varepsilon$. Supusimos que podr\'ia llega a haber alguna 
relaci\'on ya que a menor $\varepsilon$, m\'as son los n\'umeros que se consideran cero. En este esquema de implementaci\'on, que un n\'umero sea
considerado cero implica que a la hora de realizar la Eliminaci\'on Gausseana \ref{elimGaussRala}, la posici\'on no est\'a definida, por lo que 
deber\'ia realizar menos operaciones.