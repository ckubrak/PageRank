\subsubsection*{Discusión}
\paragraph{Test cuantitativos}
Los resultados de estos tests corresponden en gran medida con los esperados. Pudimos observar c\'omo aumenta notablemente el tiempo
a medida que la matriz es menos rala. Esto se debe a que cuanto menos rala es una matriz, debemos realizar m\'as operaciones y es m\'as la
memoria requerida para operar lo cual también ralentiza el proceso.\newline
Por el lado de la variaci\'on de la probabilidad también afecta al tiempo de ejecuci\'on. Tal como predijimos, a menor $p$ m\'as son
los coeficientes de la matriz $A$ lo cual puede hacer que sean "descartados" por ser su m\'odulo menor que el $\varepsilon$ utilizado.\newline
Por \'ultimo no pudimos establecer ninguna relaci\'on entre el $\varepsilon$ y el tiempo. Buscamos además
analizar el gr\'afico utlizando una escala logar\'itmica para ver si se pod\'ia apreciar algua relaci\'on, la cual no pudimos encontrar.
\paragraph{Tests cualitativos}
Los resultados obtenidos en los tests realizados con las cadenas simples, en general muestran que los nodos con mayor cantidad de caminos que pasan por ellos tienen una mayor probabilidad final, siendo esta superior cuando el nodo se halla al final del camino o antes de un nodo con estas características. La explicación puede ser que los primeros nodos de una cadena solamente van a ser accesibles por un salto aleatorio o navegando a través de links desde sus nodos precedentes. A medida que se avanza en el camino se suman las posibilidades de que algún navegante entre a ese camino y siga los links hasta los nodos que siguen. Esto se puede ver en los casos de la Cadena simple y Cadena con nodo central (caso 1 y caso 2). En la prueba Cadena con nodo central (caso 3) donde se generó un ciclo hacia el nodo que concentraba la mayor cantidad de ejes entrantes (nodo 6) y se pudo observar como los nodos 1 y 2 que ahora se pueden alcanzar por medio de un ciclo aumentan su probabilidad en comparación con los demás nodos.

En los casos de 25 nodos, los resultados obtenidos nos indican que la calidad de los links entrantes, influye en la probabilidad final de cada nodo, esto puede observarse con el nodo 13 en el caso de muchos links entrantes, ya que los nodos 1 y 25, que son aquellos con muchos links entrantes tienen apuntan a este nodo con su \'unico link de salida. Esto mismo puede observarse en el caso de muchos links salientes, debido a que el nodo 13 es donde se concentran todos los posibles caminos de salida resultantes de los nodos 1 y 25.
En los experimentos con ciclo y con ciclo al rev\'es podemos notar que el nodo 13, que act\'ua como enlace entre los dos subgrafos resulta ser de los m\'as pesados, en el caso con ciclo, los nodos m\'as pesados resultan ser los del ciclo, seguidos por el nodo 13, en el caso con ciclo al rev\'es, los nodos 22 y 20 resultan ser m\'as probables debido a que estos son aquellos nodos a donde es m\'as probable terminar, los nodos del ciclo tienen probabilidades similares, y el nodo 13 se ubica luego de estos nodos.
En el experimento de grafos separados, podemos observar que la probabilidad de cada nodo en los distintos subgrafos depende de la posici\'on que ocupan, ya sea recibiendo un \'unico link saliente de un nodo que es un paso obligado para un camino dado, o por tener un link de entrada de un nodo al cual tienen como link de salida.\\

En resumen, a cada nodo se puede acceder de dos maneras: por un salto aleatorio o por un link desde otro nodo. Cuantos más caminos alternativos existan para llegar a un nodo se espera que tenga mayor posibilidad de ser alcanzado, considerando que la probabilidad de que el navegante se mueva a través de links es mayor que la de que salte aleatoriamente. Si además el nodo no tiene muchos caminos salientes, como la probabilidad de saltos aleatorios en general es menor que la de seguir links, también puede ser un factor que contribuya a aumentar el puntaje de un nodo.
Por \'ultimo tambi\'en pudimos observar como influye el parametro $p$, una probabilidad menor tal como hab\'iamos pensado que ocurrir\'ia, nos da como resultado un ranking m\'as homogeneo, le saca peso a los v\'inculos entre nodos.