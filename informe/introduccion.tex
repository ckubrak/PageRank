\subsubsection*{Introducción Teórica}

\par El problema que se nos plantea es el de implementar un Page Rank, es decir, un método para ordernar páginas de acceso público de forma sistematizada y eficiente. Para esto tendremos que definir que parametros vamos a tener en cuenta al momento de armar nuestro órden de páginas. Estos van a ser, la cantidad de links "entrantes" a una página y la "calidad" de cada uno de estos (es decir, que tan relevante es la página de la que proviene ese link).\newline

\par Encontramos que este problema tiene una gran aplicación, dado que es fundamental para construir un motor de búsqueda y nos permitirá entender mejor como funcionan.\newline

\par HABLAR ALGO SOBRE CADENAS DE MARKOV


\par Para empezar al trabajar con matrices ralas nos vimos en la necesidad de buscar un método implementar este tipo de estructura, teniendo en cuenta un uso eficiente de los recursos del sistema (intentando sacar provecho del conocimiento que tenemos de entrada sobre las matrices). Es en este contexto que decidimos utilizar el formato CSR (Compressed Sparse Row), que vamos a proceder a explicar detalladamente más adelante.\newline

\par 