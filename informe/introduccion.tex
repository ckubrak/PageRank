\subsubsection*{Modelado del problema}
Para modelar la implementiaci\'on, utilizamos el \textit{modelo de navegante aleatorio}\cite{navegante}.
Dado una probabildad $p$ buscamos resolver el sistema:

\begin{equation}\label{eq:Axx}
    Ax = x
\end{equation}
donde 

\begin{equation}
 A = a_{ij} = \left\{
    \begin{array}{ll}
	 (1-p)/n + (p w_{i})/c & \mathrm{si\ } c_j\not=0\\
	 1/n & \mathrm{si\ } c_j=0
	 \end{array}
   \right.
\end{equation}

Como se puede ver en (citar demo), la matriz $A$ puede rescribirse como

\begin{equation}
    A = p WD + ez^{t}
\end{equation}
donde

\[
 D = d_{ij} = \left\{
    \begin{array}{ll}
	 1/c_{ij} & \mathrm{si\ } c_j\not=0\\
	 0 & \mathrm{si\ } c_j=0
	 \end{array}
   \right.
\]   
$e$ es un vector columna de dimension n y $z$ es un vector columna cuyos componentes son:
\[   
  z_{j} = \left\{
    \begin{array}{ll}
	 (1-p)/n & \mathrm{si\ } c_j\not=0\\
	 1/n & \mathrm{si\ } c_j=0
	 \end{array}
   \right.
\]  
De esta manera, la equaci\'on \ref{eq:Axx} se puede rescribir como

\begin{equation}\label{ipwd}
    (I - pWD)x = \gamma e^t 
\end{equation}
donde $\gamma = z^tx$ es el factor de escala. Para el an\'alisis de este m\'etodo supusimos
$\gamma = 1$


\subsubsection*{Introducción Teórica}

\par El problema que se nos plantea es el de implementar un Page Rank, es decir, 
un método para ordernar páginas de acceso público de forma sistematizada y eficiente. 
Para esto tendremos que definir qu\'e parametros vamos a tener en cuenta al momento de 
armar nuestro orden de páginas. Estos van a ser la cantidad de links \textit{entrantes} 
a una página y la \textit{calidad} de cada uno de estos, es decir, qu\'e tan relevante 
es la página de la que proviene ese link.\newline


\par Dado el volmunen de paginas que existen actualmente en Internet, resulta de vital importancia 
optimizar de alguna manera tanto la forma de almacenar
como la forma de procesar la informaci\'on. Para lograr dicha optimizaci\'on decidimos utlizar matrices ralas
 (d\'onde muchos de sus elementos son ceros)
\par Para empezar a trabajar con matrices ralas nos vimos en la necesidad de buscar un método implementar
 este tipo de estructura, teniendo en cuenta un uso eficiente de los recursos del sistema 
 (intentando sacar provecho del conocimiento que tenemos de entrada sobre las matrices). 
Es en este contexto que decidimos utilizar un h\'ibrido entre el formato LL (List of Lists) y 
DOK (Dictionary of Keys).\newline
