\subsubsection*{Introducción Teórica}

\par El problema que se nos plantea es el de implementar un Page Rank, es decir, un método para ordernar páginas de acceso público de forma sistematizada y eficiente. 
Para esto tendremos que definir qu\'e parametros vamos a tener en cuenta al momento de armar nuestro orden de páginas.
Estos van a ser la cantidad de links \textit{entrantes}" a una página y la \textit{calidad}" de cada uno de estos, es decir, qu\'e tan relevante es la página de la que proviene ese link.\newline

% \par Encontramos que este problema tiene una gran aplicación, dado que es fundamental para construir un motor de búsqueda y nos permitirá entender mejor como funcionan.\newline

\par HABLAR ALGO SOBRE CADENAS DE MARKOV

\par Dado el volmunen de paginas que existen actualmente en Internet, resulta de vital importancia optimizar de alguna manera tanto la forma de almacenar
como la forma de procesar la informaci\'on. Para lograr dicha optimizaci\'on decidimos utlizar matrices ralas (donde muchos de sus elementos son ceros)
\par Para empezar a trabajar con matrices ralas nos vimos en la necesidad de buscar un método implementar este tipo de estructura,
teniendo en cuenta un uso eficiente de los recursos del sistema (intentando sacar provecho del conocimiento que tenemos de entrada sobre las matrices). 
Es en este contexto que decidimos utilizar un h\'ibrido entre el formato LL (List of Lists) y DOK (Dictionary of Keys).\newline
